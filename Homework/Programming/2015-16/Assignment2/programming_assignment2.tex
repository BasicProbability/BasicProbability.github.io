\documentclass[11pt, leqno, a4paper]{article}
\usepackage{hyperref}
\hypersetup{colorlinks=true, urlcolor=blue, breaklinks=true}

\title{Programming Assignment 2 -- Basic Probability, Computing and Statistics 2015 \\[2mm]
\large{Fall 2015, Master of Logic, University of Amsterdam}}
\author{}
\date{Submission deadline: Monday, September 14th, 9 a.m.}

\begin{document}
\maketitle

\paragraph{Note:} if the assignment is unclear to you or if you get stuck, do not hesitate to contact \href{mailto:P.Schulz@uva.nl}{Philip}.

\section{Exercise}
This week we will start to work with some real data. The website \href{https://www.gutenberg.org/}{Project Gutenberg} provides access to thousands of
books which have run out of copyright. They are freely downloadable for you to read. Our goal is to analyse these books for some of their basic 
properties. In particular, we want to find their most frequent and least frequent words and also find the counts for specific words. For example,
we might want to know how often the word \textit{Alice} occurs in Alice in Wonderland.

Just as last week we will again write an interactive program. This time, our program will be able to read text files from the command line. 
Moreover, it will keep interacting with the user until the user explicitly tells it to stop. Notice that during development you can experiment
with any book you like. However, when grading we will all use \href{https://www.gutenberg.org/cache/epub/2600/pg2600.txt}{this version of War and Peace}.

As you will notice, the file contains some document information from Project Gutenberg. This means that our word counts will not be entirely correct as
there are also words that do not belong to the main text. We will not be bothered by this, however.

One last remark: when you download any book of your choice, please make sure to download the \texttt{Plain Text UTF-8} version of it. Otherwise your
program will not be able to read the file (at least not without some additional effort).

Here is what your program must be able to do:
\begin{itemize}
\item Read a text file upon start (the path to the text file will be provided in the code by the user)
\item Ask the user what he wants to do. The user should be able to choose an operation by typing the appropriate number:
\begin{enumerate}
\item Look for the most frequent words
\item Look for the least frequent words
\item Look for the frequency of a specific word
\item Exit
\end{enumerate}
\item After each operation has been finished, the program should show the user the initial choice again and be able to perform further operations until the
user decides to exit. Notice that reading the text anew for every operation would be extremely wasteful. You should try to get your program to memorize all
relevant information.
\end{itemize}

\section{Grading}
Your program counts as executable if it starts up without having a path to the text file specified on the command line. In that case the program should
explicitly ask the user to supply one. 

The task is to count words regardless of casing. This means that $ The $ and $ the $ should be treated as the same word. To achieve this, all words should be lowercased before counting them. Notice also that 
punctuation may influence the counts. Python would count $ Alice $ and $ Alice? $ as different words.
To avoid this effect, all punctuation should be removed before counting.

\begin{itemize}
\item[1 point] It is possible to enter a path to a file in the program code that will create a stream.
\item[1 point] The program gracefully shuts down if the supplied path does not exist. This means that the program prints a message informing the user
that it will exit because the file was not found. It does not raise an error!
\item[2 points] The program reads the text file only once and stores all relevant information before showing the user any options. This has to
be checked analytically by inspecting the code. Depending on your computer reading a file can take several seconds. This means that if the program was to read the file repeatedly, it might become very slow.
\item[1 points] Whenever the user enters an invalid value, the program responds ``Sorry, I do not understand this. Please choose again.'' and lets the
user re-enter the value. (As last week, user input should be either strings or numbers, depending on the question.)
\item[2 points] When the user asks for a list of most frequent words, the program responds by asking for the number $ n $ of most frequent words to 
produce. It then outputs the $ n $ most frequent words to the screen as a list in which each line looks as follows:
$$ word~~~wordFrequency $$
If there are several words of the same frequency, they are ordered alphabetically. For better readability
you should insert a tab in between the word and its frequency.
\item[2 points] When the user asks for a list of least frequent words, the program responds by asking for the number $ n $ of least frequent words to 
produce. It then outputs the $ n $ least frequent words to the screen as a list in which each line looks as follows:
$$ word~~~wordFrequency $$
If there are several words of the same frequency, they are ordered alphabetically.  For better readability
you should insert a tab in between the word and its frequency.
\item[1 point] The frequency of specific words is printed in the same format. If a word does not occur in the text, the program prints \textit{Sorry, this
word does not occur in the text}.
\item[-2 points] If the program crashes at any point while testing.
\item[-1 point] If the counting routine of the program is case-sensitive (e.g. if $ The $ and $ the $ are NOT counted as the same word).
\item[-1 point] If the program does not remove punctuation before counting. This is easy to check: just
pick a couple of words out of the text that have punctuations (e.g. $ and, $) attached to them and ask the program for their frequency. The program should tell you that those words do not occur in the text. Important: Those words should still be counted without punctuation (that is $ and, $ should ultimately
contribute to the count of $ and $).
\end{itemize}

Notice: for the frequencies, lists of the 1000 most frequent and 1000 least frequent words will be provided to you during the review period. We will also give
you detailed instructions on how to compare two lists. To test the specific word functionality, you can just pick a couple of words from these lists.

\enlargethispage{1cm}
\section{Counters}
For this exercise, it may be helpful to use a Python data structure known as 
\href{https://pymotw.com/2/collections/counter.html}{Counter}. It is not strictly necessary to use counters
for this exercise but they might make your life a lot easier. Also make sure to the check the
\href{https://docs.python.org/2/library/collections.html#collections.Counter}{official documentation}.


\end{document}