\documentclass[11pt, leqno, a4paper]{article}
\usepackage{hyperref, amsmath}
\hypersetup{colorlinks=true, urlcolor=blue, breaklinks=true}

\newcommand{\supp}{\operatorname{supp}} 
\newcommand{\E}{\mathbb{E}}

\title{Programming Assignment 4 -- Basic Probability, Computing and Statistics 2015 \\[2mm]
\large{Fall 2015, Master of Logic, University of Amsterdam}}
\author{}
\date{Submission deadline: Monday, September 28st, 9 a.m.}

\begin{document}
\maketitle

\paragraph{Note:} if the assignment is unclear to you or if you get stuck, do not hesitate to contact \href{mailto:P.Schulz@uva.nl}{Philip}.

\section{Logprobs and sampling}

\subsection{Logprobs}
When implementing probabilistic computations, one often faces several practical problems. One of those 
problems is that in realistic applications, the probabilities involved tend to get really small. Since
a computer has limited memory, it can only represent floating point
numbers up to a certain degree of precision
(i.e.\ up to a specific number of decimal digits). If a probability becomes so small that it cannot
be captured by a floating point number within the precision range of the computer, that probability will
simply be turned into 0 (this problem is known as underflow). 
This is undesirable of course. If the zeroed probability is part of a bigger 
multiplication (e.g.\ one based on the chain rule), the whole computation will yield 0 as a result. 

To prevent underflow, one often uses the logarithm of probabilities (a.k.a. logprobs). This also has an
additional advantage: multiplications in real space are additions in log space. Thus, the chain rule
can be rewritten as a sum. Since addition on a computer is much faster than multiplication, this results
in a speed-up of the computation. Incidentally, mathematicians often use logprobs because additions
are easier to handle in proofs than multiplications.

Let us recall what a logarithm (with base $ b $) is:
\begin{equation}
\log_{b}(x) \mbox{ is the number $y$ such that } b^{y} = x \, .
\end{equation}
If we omit the base $b$, we are referring to the \emph{natural}
logarithm with base $e$.

Let $ p $ and $ q $ be probabilities such that
$ p \geq q $ and define $ a := \log(p) $ and $ b := \log(q) $. Then,
the product can be computed as $\log(p \cdot q) = \log(p) + \log(q)$.
We can derive addition of logprobs as follows: 
\begin{align}
\log(p + q) &= \log \left( e^{\log(p)} + e^{\log(q)} \right) = \log \left( e^{a} + e^{b} \right)
= \log \left(e^{b} ( e^{a-b} + 1) \right) \\ 
&= b + \log \left( e^{a-b} + 1 \right) 
=\log(q) + \log (\left( e^{\log(p)-\log(q)} + 1 \right) \nonumber \ .
\end{align}

You can derive yourselves what this would look like for subtraction.

\subsection{Inverse Transform Sampling}
Another problem of probabilistic computations is sampling from a distribution. Many programming languages
come with build-in functions that allow you to sample from a \textbf{uniform} distribution. However,
in most interesting cases, you would like to sample from distributions that are non-uniform. This can
be achieved using 
\href{https://en.wikipedia.org/wiki/Inverse_transform_sampling}{inverse transform sampling}. The idea of 
inverse transform sampling is simple. To sample values of a random variable $ X $, do the following:
\begin{enumerate}
\item Sample a threshold value $ t $ uniformly in [0,1] (crucially, this is where the randomness of the 
sample comes from).
\item Evaluate the cdf of your distribution at each possible value of $ \supp(X) $.
\item As soon as the cdf is greater or equal to $ t $ return the value on which you just evaluated the cdf.
\end{enumerate}

Care has to be taken with this process. The build-in function of many programming languages only
sample in [0,1). In that case, the cdf needs to strictly greater than $ t $ before you return a value (try
to justify why this is so).

\section{The task}
Your task is to implement utility functions that do computations with logprobs and a class that 
implements the binomial distribution. For this purpose, we
supply you with \href{https://docs.python.org/2/library/unittest.html}{unit tests}. 
Your implementation is correct if it passes all tests. For details on
what each function should do, see their docstring. Notice also that some functions have comments
outside their docstrings. These are programming instructions that YOU HAVE to follow. If you do
not adhere to these instructions, points will be deducted.

Let us explain a bit more about the binomial distribution class. Given its parameters, your
class should be able to return the probability of any sequence of size $ n $. Furthermore,
it should implement a method \texttt{sample()} that returns a sequences of size $ n $. For further 
functionality, check the docstrings.

Notice that both the logarithms utility module and the binomial class may be extremely helpful in your later
programming career. You should keep them after this course has finished. Since they have been unit-tested,
you can also be fairly sure that they work correctly.

The log-probs function can be used over and over again. The binomial itself may not be of direct use but
it gives you a good intuition for how you should implement probability distributions.

The unit test can be downloaded 
\href{https://github.com/BasicProbability/PythonCode_Fall2015/tree/master/test/week4_debugging_and_testing}{here}. The skeletons of the modules that you need to implement are 
\href{https://github.com/BasicProbability/PythonCode_Fall2015/tree/master/src/week4_debugging_and_testing}{here} (\texttt{logarithms.py} and \texttt{distributions.py}). In order for the unit tests to work correctly, it is important that you maintain the file
hierarchy as you find it on Github. To this end, it may actually be easier to just clone the entire 
code folder by typing on one line
\begin{center}
\texttt{git clone \url{https://github.com/BasicProbability/PythonCode_Fall2015.git}}
\end{center}
and then work in there. This can be done by importing the cloned folder into your eclipse workspace.
Obviously, cloning will only work if you have git installed.

\section{Grading}
This time you will be very restricted in that we only allow you to use the imports and methods that
we have pre-defined for you. You may not define your own functions, methods or classes! Your code
is correct if it passes all unit tests.

\begin{itemize}
\item[0.5 points] for each passed ValueError test (3 points in total).
\item[1 point] if test\_log\_add passes.
\item[1 point] if test\_log\_difference passes.
\item[1 point] if test\_compute\_probability passes.
\item[1 point] if test\_sample\_with\_k\_successes passes.
\item[1 point] if test\_sample passes.
\item[1 point] if test\_sample\_list passes.
\item[1 point] if all the programming instructions (in-line comments) have been followed. Give 0 points here
if one or more of them have been violated.
\item[-1 point] for each additional import, class, method or function introduced by the student.
\end{itemize} 

\end{document}