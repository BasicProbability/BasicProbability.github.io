\documentclass[11pt, leqno, a4paper]{article}
\usepackage{hyperref}
\hypersetup{colorlinks=true, urlcolor=blue, breaklinks=true}

\title{Programming Assignment 1 -- Basic Probability, Computing and Statistics 2015 \\[2mm]
\large{Fall 2015, Master of Logic, University of Amsterdam}}
\author{}
\date{Submission deadline: Monday, September 7th, 9 a.m.}

\begin{document}
\maketitle

\paragraph{Note:} if the assignment is unclear to you or if you get stuck, do not hesitate to contact \href{mailto:P.Schulz@uva.nl}{Philip}.

\section{Exercise}
This week we are going to build a first small interactive program. The program is a form of dialogue system. It should ask the user questions,
store the answer and make inferences about the user based on his answers. Below, we show the system's desired ouput. 
\begin{itemize}
\item \textit{Hello, I am} some name of your choice. \textit{What is
    your name? Please only give me one first name followed by your
    entire last name (e.g. \emph{Edward Snowden} or \emph{Vincent van Gogh}.) My favourite number
    is \emph{a random number between 1 and 1000 (both inclusive)}.}
\item \textit{It's a pleasure to meet you. What is your birthday}? (specify input format here)
\item \textit{Do you smoke? (yes/no)}
\item \textit{It was nice talking to you. Let me summarise what I learnt about you. Your given name is} the given name. \textit{Your surname is}
the surname. \textit{You are} correct age \textit{years old.} 
\textit{Smoker: yes/no.} (depending on the information given by the user).
\end{itemize}

Notice that in order to compute the correct age, you need to know what date it is today. Luckily
Python has some \href{https://docs.python.org/3/library/datetime.html}{built-in functions} to help
you figure this out. You also need to find out how to generate random numbers. While you can implement
this any way you like, you may find \href{https://docs.python.org/3/library/random.html}{this documentation} helpful.

\section{Grading}
This section tells you what to look out for when programming. It also tells you how you should grade the assignment. In general, whenever you grade 
someone else's program, try to break it! This means that you should test limit cases in which the program might fail despite working correctly
on more common inputs. However, since we are at the start of this course, you should only provide input
to the program that is valid (for example, only provide your birthday in the specified format).

In the present case you should first test the program with true information about you but then also with
information that might be true for other people. For example, identify yourself once as a smoker and once as a non-smoker. Also try to provide a birthday that took place before the year 0.

As a general rule, if you the program cannot perform one of the tasks below because it does not start up
or crashes on \textbf{valid} user input, award 0 points for that task.

\enlargethispage{1cm}

\begin{itemize}
\item[1 point] The favourite number of the program is truly random. This means that it should change
almost always when you start the program.
\item[1 point] The favourite number of the program always lies in [1,1000]. This can be checked by inspecting the code.
\item[2 points] All questions show up correctly and in the right order (give full credit when the program output is formulated slightly differently from the one here)
\item[2 points] The program correctly splits up the given name and surname. Check whether it can deal with complex surnames. A friend of mine is called
Leria de la Rosa. The program should correctly identify such a surname.
\item[2 points] The program determines the correct age of the person at the time of execution. Play around with that. 1 point is to be deducted if the
program cannot deal with people born b.c. (which we will denote by negative integers) or in the year 0.
\item[1 point] The program correctly identifies smokers and non-smokers. Crucially, it should not matter whether the answer is typed in upper or lower case.
\end{itemize}


\end{document}
