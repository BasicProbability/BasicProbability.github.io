\documentclass[11pt, leqno, a4paper]{article}
\usepackage{hyperref, amsmath, amssymb}
\hypersetup{colorlinks=true, urlcolor=blue, breaklinks=true}

\title{Programming Assignment 4 -- Basic Probability, Computing and Statistics 2016 \\[2mm]
\large{Fall 2016, Master of Logic, University of Amsterdam}}
\author{Instructors: Philip Schulz, Christian Schaffner and Bas Cornellisen}
\date{Submission deadline: Wednesday, November 30th, 8 p.m.}

\begin{document}
\maketitle

\paragraph{Note:} if the assignment is unclear to you or if you get
stuck, do not hesitate to ask in the
\href{https://www.moodle.ch/lms/mod/forum/view.php?id=1721}{forum}.

\section{Assignment}

This week we will implement a Na\"ive Bayes (NB) classifier for text data. If you do not remember what a 
NB classifier is, check \href{https://github.com/BasicProbability/LectureNotes/blob/master/chapter4/chapter4.pdf}{here}. 
We will call the variable whose value we try to infer ($ Y $ in the script) the class, label or
category (all three are equivalent). The data point $ x_{i} $ are generally called \textbf{features}.
Here, we take our features to be words.
We will first train the classifier using MLE and then make predictions based on our estimates.

\subsection{Data}

We will use the famous \href{http://qwone.com/~jason/20Newsgroups/}{20 newsgroups data set} for this 
exercise. As the name suggest, there are 20 groups of text between which we have to distinguish. This
implies a baseline performance achieved by randomly choosing a group of $ 5\% $. The data are emails
within newsgroups collected in the 1990's. For a start, we will simply use all strings separated by
whitespace as features for our NB model. We call these strings words.

\subsection{Smoothing}

Many words will only occur in one class $ c $. When we see a text from that class during prediction time, the
other classes will not have probabilities for as many words as $ c $. There are two ways to go about this:
\begin{enumerate}
\item Simply ignore the words that have 0 probability under a class. This will lead to many classes
having fewer terms in the NB product than $ c $. Thus, these classes will have a higher posterior.
\item Do model all words for all classes. Unfortunately, most unseen texts (the ones we are dealing
with at prediction time) contain for each class at least one word that has 0 probability. In effect, the
posterior probability for all classes will be 0. Notice that this means the posterior is not even defined
as it does not sum up to 1.
\end{enumerate} 

As you can see, both ways will give us the wrong results. There is a cleverer way, however. We remember
all words that we have encountered during training. These define our vocabulary. During prediction we
will ignore all words not in the vocabulary. Moreover, we perform smoothing. This means that, before
doing MLE, we add a constant count for \textit{all} words in the vocabulary to the training counts
of every class. This has the effect that all vocabulary words will have positive probability under
all classes. 

\subsection{Logprobs}

Last week we mentioned that in probabilistic calculations, a product of probabilities can become so
small that it cannot be represented by your computer anymore. This is exactly the case with Na\"ive
Bayes. When you classify a document, you have to compute a potentially large product of probabilities.
To avoid turning them to 0 (and to make computation more efficient), we will work with log-probs
instead. Make sure to transform your MLEs to log-probs before doing any predictions.

\subsection{Your Task}
We have implemented a commandline interface and a skeleton of the Naive Bayes class for you. Both can be 
found \href{https://github.com/BasicProbability/BasicProbability.github.io/raw/master/Homework/Programming/2016-17/Assignment4/prerequisites.zip}{here}. Please do not manipulate \texttt{naive\_Bayes\_classifier.py}. Only work on 
\texttt{Naive\_Bayes.py}. Feel free to add any methods and data structures that you deem necessary.

\paragraph{Challenge yourself} On the development set, which we provide together with the code, you
should achieve an accuracy of $ 80.95\% $ if the smoothing constant is set to 1. Can you do better
than that? Try different smoothing constants or try to change the words (e.g. by lowercasing them) or
try to include more features (characters, for example). The possibilities are limitless! 

\subsection{Running the Code}
You will have to supply commandline arguments to the script this time. This can be done in Pycharm. 
Right-click on the run symbol and select \textit{Add parameters}. In the field \textit{script parameters}
you then have to enter the following for \texttt{naive\_Bayes\_classifier.py}
\begin{center}
\texttt{--training-corpus-dir <Path to 20news-18828> --test-set-directory <Path to dev-set>}
\end{center}
(Note: the line break was inserted by \LaTeX and is not needed in Pycharm.)
The output will be a file called \texttt{predictions.txt}. You can check your performance using the
accuracy checker. For that, you will have to pass argument to \texttt{accuracy\_checker.py}, as well. These
are simply
\begin{center}
\texttt{<Path to predictions.txt> <Path to dev\_keys.txt>}
\end{center} 

\section{Grading}
You will be graded on a test set that we will make available during peer review. The test set
is structured like the dev set but contains different files.
\begin{itemize}
\item[2 points] The code produces predictions that are in the correct format and assign labels that
it has been trained on (instead of arbitrarily named labels)
\item[2 point] The test-set (not dev-set(!)) accuracy is at least $ 40\% $.
\item[2 point] The test-set (not dev-set(!)) accuracy is at least $ 60\% $.
\item[2 point] The test-set (not dev-set(!)) accuracy is at least $ 70\% $.
\item[1 point] The test-set (not dev-set(!)) accuracy is at least $ 80\% $.
\item[1 point] The test-set (not dev-set(!)) accuracy is higher than $ 80\% $.
\end{itemize}

\end{document}