\documentclass[11pt, leqno, a4paper]{article}
\usepackage{hyperref}
\hypersetup{colorlinks=true, urlcolor=blue, breaklinks=true}

\title{Programming Assignment 1 -- Basic Probability, Computing and Statistics 2015 \\[2mm]
\large{Fall 2015, Master of Logic, University of Amsterdam}}
\author{}
\date{Submission deadline: Monday, September 7th, 9 a.m.}

\begin{document}
\maketitle

\paragraph{Note:} if the assignment is unclear to you or if you get stuck, do not hesitate to contact \href{mailto:P.Schulz@uva.nl}{Philip}.

\section{Exercise}
This week we are going to build a first small interactive program. The program is a form of dialogue system. It should ask the user questions,
store the answer and make inferences about the user based on his answers. Below, we show the system's desired ouput. 
\begin{itemize}
\item \textit{Hello, I am} some name of your choice. \textit{What is
    your name? Please only give me one first name followed by your
    entire last name (e.g. \emph{Edward Snowden} or \emph{Vincent van Gogh}.)}
\item \textit{It's a pleasure to meet you. What is your birthday}? (specify input format here)
\item \textit{Do you smoke? (yes/no)}
\item \textit{Let me test your math skills. What is number1$ *
    $number2?} Here number1 and number2 should be \textbf{random} integers between 1 and 1000
(both inclusive). This means that each time you run the program, this question will look slightly different. We have not discussed random
numbers in class. Figure out yourself how to get them.
\item Depending on the correctness of the answer, the program should say one of the following:
\begin{itemize}
\item \textit{Great, this is correct!}
\item \textit{You got that one wrong. The correct answer is} correct answer. \textit{Seems like you need to practice some more.}
\end{itemize}
\item \textit{It was nice talking to you. Let me summarise what I learnt about you. Your given name is} the given name. \textit{Your surname is}
the surname. \textit{You are} correct age \textit{years old.} 
\textit{You are a smoker/non-smoker} (depending on the information given by the user). \textit{You know how to do multiplication/You still
need to practice multiplication.} (depending on whether the math exercise was solved correctly)
\end{itemize}

\section{Grading}
This section tells you what to look out for when programming. It also tells you how you should grade the assignment. In general, whenever you grade 
someone else's program, try to break it! This means that you should test limit cases in which the program might fail despite working correctly
on more common inputs. In the present case this means that you should first test the program with true information about you but then also with
information that might be true for other people. For example, identify yourself once as a smoker and once as a non-smoker. Also solve the
math exercise correctly once and botch it the other time.

\enlargethispage{1cm}

\begin{itemize}
\item[2 points] All questions show up correctly and in the right order (give full credit when the program output is formulated slightly differently from the one here)
\item[1 point] The program correctly splits up the given name and surname. Check whether it can deal with complex surnames. A friend of mine is called
Leria de la Rosa. The program should correctly identify such a surname.
\item[2 points] The program determines the correct age of the person at the time of execution. Play around with that. 1 point is to be deducted if the
program cannot deal with people born b.c. (which we will denote by negative integers) or in the year 0.
\item[1 point] The program correctly identifies smokers and non-smokers. Crucially, it should not matter whether the answer is typed in upper or lower case.
\item[1 point] The math exercise changes its numbers at each execution.
\item[1 point] The numbers in the math exercise are indeed in the range [1,1000] and not in [0,999]. (This can be checked through a lot of repetition, but 
it might be better to do it analytically by inspecting the code). 
\item[1 point] The program correctly computes the correct solution to the exercise.
\item[1 point] The program correctly identifies whether the solution to the math exercise given by the user is wrong or right.
\item[-2 points] If the program is executable but crashes on any of the user inputs.
\end{itemize}


\end{document}
