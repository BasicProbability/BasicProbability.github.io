\documentclass[11pt, leqno, a4paper]{article}
\usepackage{hyperref, amsmath}
\hypersetup{colorlinks=true, urlcolor=blue, breaklinks=true}

\newcommand{\supp}{\operatorname{supp}} 
\newcommand{\E}{\mathbb{E}}

\title{Programming Assignment 5 -- Basic Probability, Computing and Statistics 2015 \\[2mm]
\large{Fall 2015, Master of Logic, University of Amsterdam}}
\author{}
\date{Submission deadline: Monday, October 5th, 9 a.m.}

\begin{document}
\maketitle

\paragraph{Note:} if the assignment is unclear to you or if you get stuck, do not hesitate to contact \href{mailto:P.Schulz@uva.nl}{Philip}.

\section{Na\"ive Bayes for Text Classification}

This week you will achieve two of the big goals of this course. You will implement your first machine learning 
and run it on a real data set. The algorithm will be na\"ive Bayes. You will have to implement parameter
learning and prediction. 

\paragraph{Data set} The data set is the 20 new groups data set. It consists of roughly 1000 post for each
of 20 newsgroups. The posts are of varying length but usually do not exceed the size of an e-mail. The
data set is a classic in text classification and much research had been done with it in the beginning and
mid 2000s. Nowadays it is considered too small and not very representative (after all, you want to learn
to classify text beyond what people talk about in the 20 newsgroups.