\documentclass[11pt, leqno, a4paper]{article}
\usepackage{hyperref, amsmath}
\hypersetup{colorlinks=true, urlcolor=blue, breaklinks=true}

\newcommand{\supp}{\operatorname{supp}} 
\newcommand{\E}{\mathbb{E}}

\title{Programming Assignment 5 -- Basic Probability, Computing and Statistics 2015 \\[2mm]
\large{Fall 2015, Master of Logic, University of Amsterdam}}
\author{}
\date{Submission deadline: Monday, October 5th, 9 a.m.}

\begin{document}
\maketitle

\paragraph{Note:} if the assignment is unclear to you or if you get stuck, do not hesitate to contact \href{mailto:P.Schulz@uva.nl}{Philip}.

\section{Na\"ive Bayes for Text Classification}

This week you will achieve two of the big goals of this course. You will implement your first machine 
learning algorithm and run it on a real data set. The algorithm will be na\"ive Bayes. You will have to 
implement parameter learning by maximum likelihood estimation and prediction. 

\paragraph{Data set} The data set is the 20 new groups data set. It consists of roughly 1000 post for each
of 20 newsgroups. The posts are of varying length but usually do not exceed the size of an e-mail. The
data set is a classic in text classification and much research had been done with it in the beginning and
mid 2000s. Nowadays it is considered too small and not very representative (after all, you want to learn
to classify text beyond what people talk about in the 20 newsgroups.

We have pre-processed the data for you. We removed all punctuation and also split it up into a training
set, a development set and a test set. This split is standard in machine learning. The training set contains
the largest bulk of data and you use it to train your algorithm. The development set is usually rather
small and allows you to check your algorithms performance on unseen data while developing. The test set
is the final criterion of success. In a research paper you report your results on the test set. It is meant
to represent new, unseen data. This means that you should not run you algorithm on it till the very end.

You can find the training data here and the dev set here. The test set will only be made available next week
to simulate a situation in which you really do not know the data on which you are going to apply your 
algorithm.

\paragraph{Implementation} When implementing na\"ive Bayes, we have to take care of two things:
\begin{enumerate}
\item Most documents in the test and development sets will contain words that our classifier has never
seen before. Hence, the maximum likelihood estimate for 
\end{enumerate}