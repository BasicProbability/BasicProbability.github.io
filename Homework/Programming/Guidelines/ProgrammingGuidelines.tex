\documentclass[a4paper, leqno, 11pt]{article}

\title{General guidelines for the programming exercises \\
Basic Probability statistics and computing \\[2mm]
\large{Fall 2015, Master of Logic, University of Amsterdam}}
\author{Philip Schulz \& Christian Schaffner}
\date{last modified: \today}

\begin{document}

\maketitle

\section{How grading will be done}
You will receive your grades through peer assessment. Each student will receive reviews from three other students and the points for an assignment
will be the average of the points given by the reviewers. Students can receive at most 10 points per assignment. In addition,
each student will receive an assessment of their peer grading efforts from the instructors once during the course. To this end, a couple of students will
be chosen randomly each week whose reviews will be double checked by the instructors. 

\section{Why peer grading?}
The idea of peer grading is very simple. By assessing other peoples work you learn to a) read and understand other people's code 
b) give feedback in a constructive manner c) get inspiration out of other people's work. The third point may be the most important
to your personal development as a programmer. Seeing other people's code can be of tremendous help especially when you have just
started programming. You basically get to pick and choose programming patterns from different authors that you can incorporate
into your own coding style or avoid if you find that they do not work for you.

\section{Rules of peer grading}
To make peer grading a pleasant and educating experience for all of us, here are some general rules on what you should and should
not include in your reviews.
\begin{itemize}
\item NEVER include any information in your review that could be used to identify you as the reviewer.
\item Give verbal feedback! Just deducting points won't be helpful to the person you are reviewing. Try to be precise about why
you deduct points and explain what the person could have done better.
\item Even if you give full points, try to suggest improvements. For example, could the person have made their code shorter or more
readable? If so, tell them!
\item Try to be constructive in your criticism. Instead of writing ``This is just bad'' write something like ``This part of the code
could be made better by doing X''.
\item Be polite! Do not attack the person who wrote the code but solely comment on the code itself. Do not use any inappropriate language!
\end{itemize}

\section{Submission} \label{submission}
Each of you will receive their own url to which to upload their programming assignments. This is your personal url and it should
be kept private! There will be three folders: submissions, reviewing and feedback. The submissions folder is where you place your
completed assignments. The reviewing folder will each week contain the assignments that you have to review. Finally, the feedback
folder will contain the reviews that others have written for you.

For each assignment you will need to submit a file or folder (package). Your submission should be named
\begin{center}
assignmentX\_yourStudentNumber(.py)
\end{center}
where X is the number of the assignment and yourStudentNumber speaks for itself. The .py extension is required if you are submitting a
single Python file. If you submit a package, no extension should be attached. Luckily this is a point that you will not need 
to worry about, as PyDev takes care of this. Each Python file that you submit (this includes all files in your package) should
include the line
\begin{center}
\# author: yourStudentNumber
\end{center}
at the beginning. Please make sure to not include any information (other than your student number) that identifies you as the author.

\section{Deadlines}
The deadline for the programming assignment of week i is Monday, 9 a.m. of week i+1. The deadline for the reviews of the assignment
from week i is Monday, 9 a.m. of week i+2. We will be very strict about those deadlines. Your submission time is the time stamp in
the online folder. Notice that uploading may take some time, so make sure to upload your assignments and reviews well before the deadline.
Also notice that viewing (and of course modifying) your files may change the submission time stamp. Make sure to not touch them anymore
after the deadlines!

Of course we understand that there may always be extenuating circumstances. Just make sure to let us know well in advance. Crucially,
you need to inform us \textbf{BEFORE} the deadline has passed.

\section{Grading guidelines}
For each assignment, their will be specific grading guidelines which explicitly tell you what to give points for. Important: if these
guidelines are at any point unclear and you are unsure about how to give points, we have failed in specifying the guidelines correctly
In that case you should immediately contact Philip and ask for clarification.

When you grade, write the points and verbal feedback as comments
directly into the code. For better readability, please begin each of
your comments with
\begin{center}
\# REVIEWER: 
\end{center}
At the very end of the code file you should sum up the total number of points. 
\begin{center}
\# TOTAL POINTS: X/10 
\end{center}
where X is the sum of the points you have awarded.

To finish, here is a list of things that will automatically lead to point deduction:
\begin{itemize}
\item If your code cannot be executed in Eclipse or the command line, you automatically receive 0 points for that assignment!!!! (to be done by the reviewer) 
\item If you do not put a docstring into ALL your functions, -2 points (to be done by the reviewer)
\item If your submission does not comply with the format specified in Section~\ref{submission}, -1 point (to be done by the reviewer)
\item If your submission is late by one day, -2 points (to be done by the instructors)
\item If your submission is more than one day late you receive 0 points!!!! (to be done by instructors)
\item For each review that is late, -1 point (to be done by the instructors)
\end{itemize}

\end{document}