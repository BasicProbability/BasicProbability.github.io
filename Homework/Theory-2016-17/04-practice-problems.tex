\documentclass[a4paper,10pt,landscape,twocolumn]{scrartcl}

%% Settings
\newcommand\problemset{4}
\newcommand\deadline{Wednesday September 12th, 21:00h}
\newif\ifcomments
\commentsfalse % hide comments
%\commentstrue % show comments

% Packages
\usepackage[english]{exercises}
\usepackage{wasysym,hyperref}
\DeclareMathOperator{\Cov}{Cov}
\DeclareMathOperator{\Var}{Var}

\begin{document}

\practiceproblems

{\sffamily\noindent
This week's exercises deal with joint distributions, covariance and correlation. You only have to hand in the homework problems; these exercises are optional and for practicing only. If you have questions about them, please post them to the discussion forum and try to help each other. We will also keep an eye on that.
}
%%%%%%%%%%%%%%%%%%%%%%%%%%%%%%%%%%%%%%%%%%%%%%%%%%%


\begin{exercise}[]
	Let $X$ and $Y$ be two random variables where $X$ takes values in $\{0, 6\}$ and $Y$ in $\{-6, 0, 6\}$. Their joint distribution is given by $P_{XY}(0,5) = P_{XY}(0, -5) = \frac{2}{12}$, $P_{XY}(5,-5) = \frac{5}{12}$ and all other outcomes have probability $\frac{1}{12}$. 
	
	\begin{subex}
	Find the cumulative distribution function $F_{XY}(x,y)$.	
	\end{subex}
		
	\begin{subex}
	Find the marginal distributions $P_X$ and $P_Y$.	
	\end{subex}
	
	\begin{subex}
	Are $X$ and $Y$ independent?	
	\end{subex}
	
	\begin{subex}
	Find the covariance of $X$ and $Y$.
	\end{subex}
\end{exercise}

\begin{exercise}[]
	Let $X$ and $Y$ be two independent random variables with $\Var[X] = \Var[Y] = 1$ and $E[X] = E[Y] = 2$.
	\begin{subex}
		Calculate $\Cov(5X,6Y - 7)$.
	\end{subex}
	\begin{subex}
		Calculate $\Cov(4X + Y, 6Y-6)$	
	\end{subex}
	
	\begin{subex}
		What is $E[XY]$?	
	\end{subex}
\end{exercise}

\begin{subex}
Suppose $X$ and $Y$ are random variables with the following joint pmf. Are $X$ and $Y$ independent?
	\[
	\begin{tabular}{c|c|c|c}
			&$Y=1$	&$Y=2$	&$Y=3$\\\hline
	$X=1$	&$1/18$	&$1/9$	&$1/6$ \\\hline
	$X=2$	&$1/9$	&$1/6$	&$1/8$ \\\hline
	$X=3$	&$1/6$	&$1/18$	&$1/9$  \\\hline
	\end{tabular}
	\]
\end{subex}


\begin{exercise}[]
Let $X$ be a random variable for which $P(X\ge 0)$ and $E[X]$ are both strictly positive. Show that $P(X>0)$ is strictly bigger than $0$.
\end{exercise}

\begin{exercise}
The \emph{Poisson distribution} 	models the number of time some event happens in a given period of time. Its probability mass function is given by
\[
P(X = k) = {\frac {\lambda ^{k}e^{-\lambda }}{k!}}, \qquad \lambda \in \mathbb{R}_{>0}, \quad k=0,1,2,\dots
\]
where $\lambda$ is the distribution's only parameter.

Suppose $X \sim \text{Poisson}(\lambda_x)$ and $Y \sim \text{Poisson}(\lambda_y)$ are two independent variables. 

	\begin{subex}
	Show that $X+Y$ is $\text{Poisson}(\lambda_x + \lambda_y)$-distributed.
	\end{subex}
	
	\begin{subex}
	Calculate $\Cov(X, Y)$.
	\end{subex}
	


\end{exercise}




\end{document}