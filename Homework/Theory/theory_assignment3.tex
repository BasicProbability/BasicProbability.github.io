\documentclass{article}

\usepackage{amsmath}
\usepackage{amssymb}
\usepackage{calc}
\usepackage{fullpage}
\usepackage{hyperref}
\hypersetup{colorlinks=true, urlcolor=blue, breaklinks=true}

%\newcommand{\prop}[1]{\lbrack \! \lbrack #1 \rbrack \! \rbrack}

\newcommand{\philip}[1]{ \textcolor{red}{\textbf{Philip:} #1}}
\newcommand{\chris}[1]{ \textcolor{blue}{\textbf{Chris:} #1}}


\title{Theory Assignment 3 -- Basic Probability, Computing and Statistics\\[2mm]
\large{Fall 2015, Master of Logic, University of Amsterdam}}

\author{}
\date{Submission deadline: Monday, September 21th, 2015, 9 a.m.}



\begin{document}
\maketitle

\paragraph{Cooperation}
Cooperation among students for both theory and programming exercises
is strongly encouraged.  However, after this discussion phase, every student writes down and submits his/her own individual solution.

\paragraph{Guidelines}
	The starred exercises are relatively easy exercises for you to practice. No points are awarded for them. You may pick and choose  2 exercises from each exercise type your submission, i.e. you need to submit a total of 4 exercises to be able to get all points. Numbered exercises with an exclamation mark are supposed to be a bit harder and you may challenge yourself by trying to solve them.

In the directory of your private url there is folder called `theory\_submissions'. Please upload your submission there. Your submission should be a PDF-document (use a scanner for handwritten documents!) entitled \textit{AssignmentX\_yourStudentNumber.pdf}, where \textit{X} is the number of the assignment and \textit{yourStudentNumber} is your student number. If your submission does not comply with this format, we will deduct 1 point. For each day that your submission is late, we deduct 2 points.

If you have any question about the homework or if you need help, do not hesitate to contact \href{mailto:T.S.Brochhagen@uva.nl}{Thomas}.

\paragraph{Exercises}

\paragraph{Type I [2 exercises: 2 points per exercise]}
\begin{enumerate}
	\item Assume that 4 balls are randomly picked one after another from an urn of 20 balls, labeled from 1 to 20, without returning them. Let $X$ stand for the largest label selected. What is $P_X(X \geq 16)$?
	\item Let $X$ be a random variable taking any of the values $0$, $5$ or $-5$ with probabilities $P_X(X = -5) = .3, P_X(X = 0) = .3, P_X(X = 5) = .4$, respectively. What is $E_{P_X}[X^2]$?
	\item Compute the probability mass function of the number of tails for 7 subsequent coin tosses. 
	\item A peddler is trying to sell her goods to two different clients. She estimates her first meeting to yield a profit with probability $.4$, and her second, independently, with probability $.7$. Any profit made is equally likely to be either from selling her luxury goods, which bring a profit of \$$1000$, her generic goods, brining in a profit of \$$500$. Compute the probability mass function of $X$, where $X$ is the total value of her profits.
	\item Let $E[X] = 5$ and $var(X) = 7$, compute
		\begin{itemize}
			\item[(i)] $E[(2 + X)^2]$
			\item[(ii)] $var(4 + 3X)$
		\end{itemize}
	\item[6!] Indicate the maximal number of people you can invite to your party so that the probability of any of them having the same birthday as you is less than $\frac{1}{2}$. Assume that birthdays are uniformly distributed and that we do not care about a persons birth year. 
\end{enumerate}

\paragraph{Type II [2 exercises: 3 points per exercise]}
\begin{enumerate}
\item Calculate the variance of a loaded 6-sided die that has a probability of $\frac{1}{6}$ for all odd numbered sides, and $\mathbb{P}(2) = .1,  \mathbb{P}(4) = .1, \mathbb{P}(6) = .3$.
	\item A jar contains $N$ Euro and $M$ GBP coins. Coins are taken out randomly up to the first draw of a GBP coin. If each drawn coin is put back before picking a new one, what is the probability that
		\begin{itemize}
			\item[(i)] exactly $n$ draws are needed?
			\item[(ii)] at least $k$ draws are needed?
		\end{itemize}
	\item You make a bet with a friend to the effect that he has to pay you an amount $L$ should an event $M$ happen within a year. If you estimate $M$ to happen with probability $q$ within this period, what should you charge him to enter the bet for an expected profit of $10\%$ percent of $L$?
	\item Consider a group of $n$ randomly chosen students and let $E_{i,j}$ denote the event that students $i$ and $j$ have the same birthday, $i \neq j$. Under the assumption that the students' birthdays are uniformly distributed throughout the same year, compute
		\begin{itemize}
			\item[(i)] $\mathbb{P}(E_{c,d}|E_{a,b})$
			\item[(ii)] $\mathbb{P}(E_{a,c}|E_{a,b})$
			\item[(iii)] $\mathbb{P}(E_{b,c}|E_{a,b} \cap E_{a,c})$
		\end{itemize}
	\item Show that for any two RVs $ X $ and $ Y $ with joint distribution $ P_{XY} $ it holds that $ E[X+Y] = E[X] + E[Y] $.
	\end{enumerate}

\paragraph{Type III}
\begin{enumerate}
\item For a RV $ X \sim binom(n,\theta) $ with $ n \in \mathbb{N}, n > 0 $ and $ \theta \in [0,1] $ calculate
	\begin{enumerate}
		\item the expectation $ E[X] $ and
		\item the variance $ var(X) $.
	\end{enumerate}
	Notice that the computation should be general, i.e. neither $ n $ nor $ \theta $ should be fixed.
\end{enumerate}
\end{document}
