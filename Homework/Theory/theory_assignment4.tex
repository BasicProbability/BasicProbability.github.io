\documentclass{article}

\usepackage{amsmath}
\usepackage{amssymb}
\usepackage{calc}
\usepackage{fullpage}
\usepackage{hyperref}
\hypersetup{colorlinks=true, urlcolor=blue, breaklinks=true}

%\newcommand{\prop}[1]{\lbrack \! \lbrack #1 \rbrack \! \rbrack}

\newcommand{\philip}[1]{ \textcolor{red}{\textbf{Philip:} #1}}
\newcommand{\chris}[1]{ \textcolor{blue}{\textbf{Chris:} #1}}


\title{Theory Assignment 4 -- Basic Probability, Computing and Statistics\\[2mm]
\large{Fall 2015, Master of Logic, University of Amsterdam}}

\author{}
\date{Submission deadline: Monday, September 28th, 2015, 9 a.m.}



\begin{document}
\maketitle

\paragraph{Cooperation}
Cooperation among students for both theory and programming exercises
is strongly encouraged.  However, after this discussion phase, every student writes down and submits his/her own individual solution.

\paragraph{Guidelines}
Starred exercises are relatively easy exercises for you to practice. No points are awarded for them. You may pick and choose {\bf N exercises from exercise type I}, as well as {\bf N from exercise types II and III} from  your submission, i.e. you need to submit {\bf a total of 4 exercises} to be able to get all points. Numbered exercises with an exclamation mark are supposed to be a bit harder and you may challenge yourself by trying to solve them.

In the directory of your private url there is folder called `theory\_submissions'. Please upload your submission there. Your submission should be a PDF-document (use a scanner for handwritten documents!) entitled \textit{AssignmentX\_yourStudentNumber.pdf}, where \textit{X} is the number of the assignment and \textit{yourStudentNumber} is your student number. If your submission does not comply with this format, we will deduct 1 point. For each day that your submission is late, we deduct 2 points. N.B.: If multiple files are submitted for a single assignment before the deadline, the latest version will be graded.

If you have any question about the homework or if you need help, do not hesitate to contact \href{mailto:T.S.Brochhagen@uva.nl}{Thomas}.

\paragraph{Exercises}

\paragraph{Type I [n exercises: p points per exercise]}

\begin{enumerate}
	\item An insurance company divides clients into two groups: cautious and accident prone drivers. Based on their statistics, the probability that an accident prone client will actually be involved in an accident within a one year period is $.4$, whereas the probability for cautious drivers is $.2$. Under the assumption that $40\%$ of the population is accident prone, calculate the probability that a new client will have an accident within a year of purchasing an insurance.
	\item Many multiple-choice tests do not deduct points for wrong answers. Therefore, should one not know the answer, guessing is a recommendable strategy. Let $k$ be the probability that a student knows the answer to a multiple-choice question, and $1-k$ that he guesses instead. Further, let $\frac{1}{c}$ be the probability of guessing correctly, where $c$ is the number of alternatives in the multiple-choice question. Compute the probability that a student knew the answer to a question given that it was answered correctly.
	\item A friend has 3 cards, identical in form but not in color: the first is green from both sides, the second is yellow from both sides, and the third has one yellow and one green side. She mixes the three cards, selects one at random, and puts it down on the table. The card shows a green side. Would you rather bet money on the card's other side being yellow, green, or either? Indicate the relevant computations and reasoning used to arrive at this conclusion.
	\item A factory has 11 automatic machines that drop their produced goods into 11 unlabeled boxes. The production of the goods is highly complex and prone to error. As a consequence, ten of the machines produce one in six defective goods. Matters are worse for the eleventh machine as it produces one in three defective goods. Each of the machines' boxes is shipped to quality control with 12 goods inside. Your job at quality control is to decide whether the goods of a box were produced by the bad eleventh machine or not. To save costs only some of the box's contents are inspected. Let $G$ denote a `good' product and $B$ a `bad' one. Compute the probability of having a box from the eleventh machine when, after subsequent inspection, witnessing
		\begin{itemize}
			\item[(i)] $B$ (the first product was bad);
			\item[(ii)] $BBG$ (the first two products were bad, the third good);
			\item[(iii)] $BBGBBB$ (...).
		\end{itemize}
      
\end{enumerate}		


\end{document}
