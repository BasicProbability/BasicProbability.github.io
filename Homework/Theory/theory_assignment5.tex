\documentclass{article}

\usepackage{amsmath}
\usepackage{amssymb}
\usepackage{calc}
\usepackage{fullpage}
\usepackage{hyperref}
\hypersetup{colorlinks=true, urlcolor=blue, breaklinks=true}

%\newcommand{\prop}[1]{\lbrack \! \lbrack #1 \rbrack \! \rbrack}

\newcommand{\philip}[1]{ \textcolor{red}{\textbf{Philip:} #1}}
\newcommand{\chris}[1]{ \textcolor{blue}{\textbf{Chris:} #1}}


\title{Theory Assignment 5 -- Basic Probability, Computing and Statistics\\[2mm]
\large{Fall 2015, Master of Logic, University of Amsterdam}}

\author{}
\date{Submission deadline: Monday, October 5th, 2015, 9 a.m.}



\begin{document}
\maketitle

\paragraph{Cooperation}
Cooperation among students for both theory and programming exercises
is strongly encouraged.  However, after this discussion phase, every student writes down and submits his/her own individual solution.

\paragraph{Guidelines}
You may pick and choose {\bf N exercises from exercise type I}, as well as {\bf M from exercise type II} for submission, i.e. you need to submit {\bf a total of N+M exercises} to be able to get all points. Numbered exercises with an exclamation mark are supposed to be a bit harder and you may challenge yourself by trying to solve them.

In the directory of your private url there is folder called `theory\_submissions'. Please upload your submission there. Your submission should be a PDF-document (use a scanner for handwritten documents!) entitled \textit{AssignmentX\_yourStudentNumber.pdf}, where \textit{X} is the number of the assignment and \textit{yourStudentNumber} is your student number. If your submission does not comply with this format, we will deduct 1 point. For each day that your submission is late, we deduct 2 points. N.B.: If multiple files are submitted for a single assignment before the deadline, the latest version will be graded.

If you have any question about the homework or if you need help, do not hesitate to contact \href{mailto:T.S.Brochhagen@uva.nl}{Thomas}.

\paragraph{Exercises}

\paragraph{Type I [n exercises: p points per exercise]}

\begin{enumerate}
	\item Let $f(x) = \frac{2x + 2}{x + 1}$. What is $\lim_{x \to -1} f(x)$?
	\item Let $f: \mathbb{R} \to \mathbb{Z}, f(x) = max\{m \in \mathbb{Z} | m \leq x\}$. Explain why $\lim_{x \to 2} f(x)$ does not exist. 
	\item Compute (i) $\frac{d}{dx}(x^3 + 1)$, (ii) $\frac{d}{dx}5$, (iii) $\frac{d}{dx} x^{-100}$
	\item Assume that you sample the age (in years) of students following an online class and get the following values: 12, 18, 34, 16, 22, 15, 25, 27, 24, 16, 23, 23, 41. Compute the sample mean and its variance.
	\item In statistics, the {\em three-$\sigma$ rule of thumb} or the $68-95-99.7$ {\em rule} is a mnemonic shorthand to remember the approximate percentage of values that fall within, respectively, one, two or three standard deviations (the positive square root of the variance) of the mean in a normal distribution. More precisely, $P_X(\mu - \sqrt[+]{var(X)} \leq x \leq \mu + \sqrt[+]{var(X)}) \approx 0.6827$,  $P_X(\mu - 2\sqrt[+]{var(X)} \leq x \leq \mu + 2\sqrt[+]{var(X)}) \approx 0.9545$, and $P_X(\mu - 3\sqrt[+]{var(X)} \leq x \leq \mu + 3\sqrt[+]{var(X)}) \approx 0.9973$.\\ Use the $68-95-99.7$-rule to estimate the probability that a groundhog lives less than $3.6$ years given that the average marmot lives $2.7$ years with a standard deviation of $0.3$ years. Assume their lifespans to be normally distributed.
	\item Imagine an urn with an unknown number of red and blue balls. Assume you select one ball at a time with replacement and count the ratio of red to blue. Will a single subsequent selection always increase the accuracy of your estimate of the underlying distribution? Justify your answer.  
	\item Assume you toss a fair coin $1000$ times and the first $100$ tosses all turn out to be tails. What proportion of tails do you expect for the remainder $900$ tosses? Justify your answer by taking the Weak Law of Large Numbers into consideration. 
\end{enumerate}		

\paragraph{Type II [n exercise: p points per exercise]}
\begin{enumerate}
	\item Imagine that the number of beers sold at a bar during a day is a random variable with mean $50$.
		\begin{itemize}
			\item[(i)] Give an upper-bound for the probability that this week's sale will be more than $75$ beers
			\item[(ii)] Assume that the variance of a week's beer sale is $25$. What can we say about the probability of this week's sale being between $40$ and $60$ beers?
		\end{itemize}
	\item For $X$ with variance $\sigma^2$ and mean $\mu$, show that $P(|X-\mu| \geq k\sqrt[+]{\sigma^2}) \leq \frac{1}{k^2}$
\end{enumerate}

\paragraph{Type III Type II [n exercise: p points per exercise]}
\begin{enumerate}
	\item Assume you are given a data sample $ x $ and your model for that sample is a multinomial distribution with parameters $ n $ and 
	$ \theta_{1}, \ldots, \theta_{n} $. What is the MLE for parameter $ \theta_{i}, 1 \leq i \leq n $? \textbf{Hint:} To solve this exercise you will
	have to use partial differentiation, i.e. you should use $ \frac{\delta}{\delta\theta_{i}}\mathcal{L}_{x} $.
	\setcounter{enumi}{\value{enumi}+1}
	\item[\arabic{enumi}!] This exercise asks you to compute the MAP estimate for the binomial distribution parameter with an arbitrary beta prior. The beta distribution
	is a standard choice as a prior distribution for the binomial. It looks as follows:
	\begin{equation*}
	P(\Theta=\theta) = \dfrac{\Gamma(\alpha + \beta)}{\Gamma(\alpha)\Gamma(\beta)}\theta^{\alpha} \times (1-\theta)^{\beta}
	\end{equation*}
	Thus your task is to compute
	\begin{align*}
	\underset{\theta}{arg\, \, max}~P(\Theta=\theta|X=x) &= \underset{\theta}{arg\, \, max}~P(X =x |\Theta = \theta) \times P(\Theta = \theta) \\
	&= \underset{\theta}{arg\, \, max}~\binom{n}{k}\theta^{k}\times (1-\theta)^{n-k} \times \dfrac{\Gamma(\alpha + \beta)}{\Gamma(\alpha)\Gamma(\beta)}\theta^{\alpha} \times (1-\theta)^{\beta}
	\end{align*}
\end{enumerate}

\end{document}
