\documentclass[14pt]{beamer}

\begin{document}

\begin{frame}{Discrete Distributions}
\begin{table}
\begin{tabular}{|c||c|c|c|c|}
\hline
$X$ 				&	1	&	3	&	5	&	7	\\
\hline
$F(X \leq x)$	&	.5	&	.75	&	.9	&	1	\\
\hline
\end{tabular}
\end{table}
\begin{enumerate}[a)]
\item What is $ P(X \leq 3) $?
\item What is $ P(X = 3) $?
\end{enumerate}
\end{frame}

\begin{frame}{Answers}
\begin{enumerate}[a)]
\item 0.75
\item 0.75-0.5 = 0.25
\end{enumerate}
\end{frame}

\begin{frame}{Expectation}
\begin{enumerate}[a)]
\item Would you accept a gamble that offers a 10\% chance to win \$95 and a 90\% chance of losing \$5?
\item Would you pay \$5 to participate in a lottery that offers a 10\% chance to win \$100 and a \%90
percent chance to win nothing?
\end{enumerate}
\end{frame}

\begin{frame}{Partial Answers}
This is the same calculation twice:
$$ 0.1 \times 95 - 0.9 \times 5 = 9.5 - 4.5 = 5 $$
\end{frame}

\begin{frame}{Memorylessness}
Assume that $ X \sim Geometric(p) $. Show that the geometric distribution is memoryless (or stationary), i.e.
show that 
$$ P(X = n + k | X \geq n) = P(X = k) $$
where $ n,k > 0 $.
\end{frame}

\begin{frame}{Answer}
\begin{itemize}
\item By definition : $ P(X = n + k | X \geq n) = \frac{P(X = n + k, X \geq n)}{P(X \geq n)} $
\item We calculate $ P(X \geq n) $ as $ (1-p)^{n} $
\end{itemize}
\begin{align*}
&\frac{P(X = n + k, X \geq n)}{P(X \geq n)} = \frac{P(X = n + k)}{P(X \geq n)} \\
&= \frac{p(1-p)^{n+k}}{(1-p)^{n}} = p(1-p)^{k}
\end{align*}
\end{frame}

\begin{frame}{Variance}
\begin{table}
\center
\begin{tabular}{|c||c|c|c|c|c|}
\hline
$ X $		& 1	&	2	&	3	&	4	&	5	\\
\hline
$P(X = x)$	& .1 & .2	&	.4	&	.2	&	.1	\\
\hline 
\end{tabular}
\end{table}
\begin{enumerate}[a)]
\item Compute the variance and standard deviation $ \sigma(X) $ of $ X $.
\item What are the variance and standard deviation of $ \frac{X}{\sigma(X)} $?
\end{enumerate}
\end{frame}

\begin{frame}{Answers}
\begin{enumerate}[a)]
\item $ \mathbb{E}(X) = 3 $ and thus $ var(X) = .1 \times 4 + .2 \times 2 + .1 \times 4 = 1.2 $. It follows
that $ \sigma(X) = \sqrt{var(X)} = \sqrt{1.2} $.
\item For any real RV $ Y $ we have $ var\left(\frac{Y}{\sigma(Y)}\right) = \sigma(Y)^{-2}\sigma(Y)^{2}= 1 $. 
This is called normalisation.
\end{enumerate}
\end{frame}

\end{document}