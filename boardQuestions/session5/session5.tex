\documentclass[14pt]{beamer}

\usepackage{nicefrac, enumerate}

\begin{document}

\begin{frame}{Coin Tosses}
\small
A coin is taken from a box containing three coins, which give heads
with probability $ p = \nicefrac{1}{3} , \nicefrac{1}{2} $, and $ \nicefrac{2}{3}$. The mystery coin is 
tossed 80 times, resulting in 49 heads and 31 tails.
\begin{enumerate}[a)]
\item What is the likelihood of this data for each type on coin? Which
coin gives the maximum likelihood?
\item Now suppose that we have a single coin with unknown probability
p of landing heads. Find the likelihood and log likelihood functions
given the same data. What is the maximum likelihood estimate for p?
\end{enumerate}
\end{frame}

\begin{frame}{Coin Tosses}
\small
\begin{enumerate}[a)]
\item The data $ x $ is 49 heads in 80 tosses.
We have three hypotheses: the coin has probability
$ p = \nicefrac{1}{3}, p = \nicefrac{1}{2}, p = \nicefrac{2}{3} $. So the likelihood function $ L_{x} : p \mapsto P(X=x|P=p) $ takes 3 values:
\begin{align*}
L_{x}(\frac{1}{3}) &= \binom{80}{49}\left(\frac{1}{3}\right)^{49}\left(\frac{2}{3}\right)^{31} = 6.24 \cdot 10^{-7} \\
L_{x}(\frac{1}{2}) &= \binom{80}{49}\left(\frac{1}{2}\right)^{49}\left(\frac{1}{2}\right)^{31} = 0.024 \\
L_{x}(\frac{2}{3}) &= \binom{80}{49}\left(\frac{2}{3}\right)^{49}\left(\frac{1}{3}\right)^{31} = 0.082
\end{align*}
Thus the maximum likelihood is achieved under $ p = \nicefrac{2}{3} $.
\end{enumerate}
\end{frame}

\begin{frame}{Coin Tosses}
\begin{enumerate}[b)]
\item We already know that the MLE for the Bernoulli (and binomial) distribution is 
$ \frac{k}{n} $ where $ k $ is the number of successes and $ n $ is the total number of Bernoulli trials.
Thus we get $$ p^{*}=\frac{49}{80} = 0.6125 $$ as the MLE in the present example.
\end{enumerate}
\end{frame}

\begin{frame}{Geometric MLE}
For an i.i.d. data set $ x = x_{1}^{n} $ find the MLE for the geometric distribution:
$$ P(X=x) = (1-\theta)^{x}\theta $$
\end{frame}

\begin{frame}{Geometric MLE}
The likelihood function is
$$ L_{x}(\theta) = \prod_{i=1}^{n} \theta(1-\theta)^{x_{i}} = \theta^{n} (1-\theta)^{\sum_{i=1}^{n}x_{i}} $$
and thus the log-likelihood is
$$ \mathcal{L}_{x}(\theta) = n\log(\theta) + \sum_{i=1}^{n}x_{i}\log(1-\theta) \ . $$
\end{frame}

\begin{frame}{Geometric MLE}
In the next step we compute the score function.
\begin{align*}
\frac{d}{d\theta}\mathcal{L}_{x}(\theta) &= \frac{d}{d\theta}n\log(\theta) + \frac{d}{d\theta}\sum_{i=1}^{n}x_{i}\log(1-\theta) \\
&= \frac{n}{\theta} - \frac{\sum_{i=1}^{n}x_{i}}{1-\theta}
\end{align*}
\end{frame}

\begin{frame}{Geometric MLE}
Setting this to 0 gives
\begin{align*}
0 &= \frac{n}{\theta} - \frac{\sum_{i=1}^{n}x_{i}}{1-\theta} &\Leftrightarrow \\
\frac{n}{\theta} &= \frac{\sum_{i=1}^{n}x_{i}}{1-\theta} &\Leftrightarrow \\
n - n\theta &= \theta \sum_{i=1}^{n}x_{i} &\Leftrightarrow \\
\theta &= \frac{n}{n + \sum_{i=1}^{n}x_{i}}&
\end{align*}
\end{frame}

\end{document}