\documentclass[11p,a4paper]{article}

\usepackage{amsmath,amssymb,amsthm,enumerate,dsfont,mathpazo,xcolor}
\usepackage{hyperref,fullpage}
\hypersetup{colorlinks=true, linkcolor=blue, urlcolor=blue}

\newtheorem{theorem}{Theorem}

\title{Board Questions}
\date{Seventh Session, Oct 17, 2016}

\begin{document}
\maketitle

\section{Chain Rule for Entropy}
Prove the chain rule for entropy, namely that $H(X,Y)=H(X|Y)+H(Y)$.

\section{Data Compression}
For the rest of today, we are studying the problem of \emph{data
  compression}. Assume with have a source of information which emits
four different symbols $a,b,c,d$ with probabilities $1/2,1/4,1/8,1/8$,
respectively. We model our source as iid realisation of a
categorical random variable $X$ with distribution $P_X$. A typical
sequence of symbols from this source could look like this: $bababcdbbaabadbaaaa$.
Our task is to \emph{compress} such sequences as much as
possible. Formally, we would like to map every source symbol to a
binary string such that the average code length is minimal.


\subsection{Codes} \label{sec:codes}
The following are four (binary symbol) codes $C,D,E, F$ for the
categorical random variable $X$, with ${\cal X} = \{a,b,c,d\}$:
\[
\begin{array}{c | c | l | l | l | l}
x & P(X=x) & C(x) & D(x) & E(x) & F(x)\\
\hline
a & 1/2 & 0   & 0   & 0   & 00 \\
\hline
b & 1/4 & 10  & 010 & 01  & 01\\
\hline
c & 1/8 & 110 & 01  & 011 & 10 \\
\hline
d & 1/8 & 111 & 10  & 111 & 11 \\
\end{array}
\]

These codes can be used to encode strings of symbols by concatenation . For instance,
the encoding of string ``adba'' under code $E$ is
\[
E(adba) = E(a) E(d) E(b) E(a) = 0 \; 111 \; 01 \; 0 = 0111010
\]

\begin{enumerate}
\item What is the encoding of $adba$ under codes $C$, $D$ and $F$?
\item What is the decoding of $001001110$ under code $C$?
% note that there is some extra symbol at the end that cannot be decoded yet
\item What is the decoding of $0100100$ under code $D$? Is it unique?
% no, that code sucks so there are several ways of decoding possible
\item What is the decoding of $001111$ under code $E$? Is it unique?
  What happens if you learn that the next bit is $1$ (so you have to decode
  $0011111$ under $E$)?
% you have to revisit previous decodings, because E is not prefix-free
\item Can you prove that arbitrary concatenations of codewords of $C$
  are uniquely decodable? What about concatenations of codewords of
  $E$ or $F$?
% C is prefix-free, E is its inverse, so post-fix free, F is also
% prefix-free
\item Which of the above codes is the most convenient to work with in
  terms of encoding and decoding? Why?
% C and F are the easiest, due to prefix-freeness. $E$ would be
% easiest in the arabic world.
\end{enumerate}

\subsection{Code Length}
\newcommand{\len}{\ell}
 The \emph{average code length} of a binary symbol code is defined as follows.
Let $\len(s)$ denote the length of a string $s \in \{0,1\}^*$. The (average) length of a code $C$ for a source $X$ is defined as
\[
\len_C(X) := \mathbb{E}[\len(C(X))] = \sum_{x \in \mathrm{supp}(X)} P(X=x) \len(C(x)) \, .
\]
\begin{enumerate}
\item Compute $\len_C(X)$, $\len_D(X)$, $\len_E(X), \len_F(X)$ for the codes of the previous section.
\item Compute the entropy $H(X)$ for the distribution $P_X$ above and compare both the obtained values and the way you have obtained them.
% observe that the computations are very similar, and sometimes you
% actually up computing the entropy. In particular, if the length of a
% symbol corresponds to its surprisal value in bits.
\end{enumerate}

In the Information Theory course, we will prove Shannon's
source-coding theorem:
\begin{theorem}
Let $P_X$ be a distribution and $\ell_{\min}(X) := \min_C \len_C(X)$
the minimal average codeword length among all uniquely decodable
codes. Then,
\[
H(X) \leq \ell_{\min}(X) \leq H(X)+1 \, .
\]
\end{theorem}
In other words, the Shannon entropy pretty much determines the optimal
average codeword length.

\subsection{Optimal Codes}
\begin{enumerate}
\item Show that code $C$ from Section~\ref{sec:codes} is optimal in
  terms of average coding length.
% use the source-coding theorem
\item Construct an optimal symbol code for the following
  distribution:
\[
\begin{array}{c | c | c | c | c | c | c | c}
y & a & b & c & d & e & f & g\\
\hline
P(Y=y) & 1/4 & 1/4   & 1/8   & 1/8 & 1/8 & 1/16 & 1/16 \\
\end{array}
\]
% for instance: 00 , 01, 100, 101, 110, 1110, 1111 
{\bf Hint:} Should symbols with high probability to occur receive long or short codewords?
\item Prove that the code you found is optimal!
\item Look up on the internet what
  \href{https://en.wikipedia.org/wiki/Huffman_coding}{Huffman coding}
  is and use it to find an optimal binary symbol code for the following distribution:
\[
\begin{array}{c | c | c | c | c | c}
z & a & b & c & d & e\\
\hline
P(Z=z) & 0.25 & 0.25   & 0.2   & 0.15 & 0.15 \\
\end{array}
\]
\end{enumerate}
% 10, 00, 01, 110, 111

\subsection{Randomness-Efficient Sampling}
Let's consider a different problem, namely how to efficiently sample iid from a
distribution $P_X$. Explain how to repeatedly sample from $P_X$ given an optimal binary code
and access to uniformly distributed random bits. How many random bits
per sampled symbol do you need on average?
% decode the random bits according to the code.
% roughly H(X) bits per symbol are required

\end{document}